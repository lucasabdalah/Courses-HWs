\subsection{Problema 1 - \texorpdfstring{$M$}{M}-QAM}

Considere a modulação $M$-QAM, em que o sinal em banda base é dado por:
$$s_m(t) = ( A_m^{(\text{real})} + j A_m^{(\text{imag})}) g(t) ,$$
em que $g(t)$ é um pulso transmitido, $A_m^{(\text{real})}$ e $A_m^{(\text{imag})}$ são amplitudes da parte real e imaginária da forma de onda transmitida, respectivamente.

Considere $\int_{-\inf}^{\inf} |g(t)|^2 \,dt = \mathcal{E}_{g} = 1$, isto é, o pulto $g(t)$ possui energia unitária. Suponha a transmissão de uma sequência de símbolo $\{s_{m}\}$ de tamanho $L = 26400 \text{bits}$
\begin{enumerate}
    \item Para $M = \{ 4, 16, 64\}$, determine a energia média $\mathcal{E}_{m}$ de cada constelação;
    \item Para $M = \{ 4, 16, 64\}$, determine a distância mínima $d_{min}$ entre dois símbolos;
    \item Para $M = \{ 4, 16, 64\}$, implemente o modulador (mapeamento bit-símbolo) usando a codificação de Gray;
    \item Para $M = \{ 4, 16, 64\}$, implemente o demodulador (mapeamento símbolo-bit).
\end{enumerate}

% -------------------------------------------------------------------

\subsubsection{Energia da Constelação} 

O desenvolvimento é citado em~\cite{Proakis, Cecilio}.

\begin{table}[!ht]
    \centering
    \begin{tabular}{|c|c|c|c|}
    \hline
    $\mathcal{E}_{media}$ & $\mathcal{E}_{media(bit)}$ & $d$ \\ \hline
     &  &  \\ 
     $\frac{M-1}{3} \mathcal{E}_g$ & $ \frac{M-1}{3\log_2 M} \mathcal{E}_g$ & $\sqrt{\frac{3 \mathcal{E}_{media}}{2(M-1)}} $ \\ 
     &  &  \\ \hline
    \end{tabular}
    \caption{Frequência da onda de entrada e a tensão máxima da saída do circuito integrador.}
    \label{tab:QAM}
\end{table}

% -------------------------------------------------------------------
\subsubsection{Distância Mínima entre Símbolos}

Como calcular os coeficiente para constelação $M$-QAM retangular, onde $\sqrt{M}$ assume valores inteiros. Os coeficientes em quadratura $a_i$ e $b_i$ são obtidos através da equação: $\{ (2i -\sqrt{M} - 1)d \}_{i=1}^{\sqrt{M}} $ 

A distância eucliadiana entre os sinais na modulação QAM é
$$ d_{mn} = \sqrt{|| s_m - s_n||^2}$$ 
$$ = \sqrt{\frac{\mathcal{E}_g}{2}[(A_{mi} - A_{ni})^2 + (A_{mq} - A_{nq})^2]}$$

\subsubsection{Modulador}

\begin{figure}[!ht]
    \centering
    \includegraphics[width=1.0\textwidth,clip=true,trim={1.5cm 8.5cm 1.8cm 8.3cm}]{C:/Users/lukin/Documents/GitHub/Courses-HWs/Sistemas de Comunicacoes Digitais/matlab/problema1/fig/4_QAM_plot.pdf}
    \caption{Exemplo de 4-QAM plot.}
    \label{fig:4_QAM_plot}
\end{figure}

\begin{figure}[!ht]
    \centering
    \includegraphics[width=1.0\textwidth,clip=true,trim={1.5cm 8.5cm 1.8cm 8.3cm}]{C:/Users/lukin/Documents/GitHub/Courses-HWs/Sistemas de Comunicacoes Digitais/matlab/problema1/fig/16_QAM_plot.pdf}
    \caption{Exemplo de 16-QAM plot.}
    \label{fig:16_QAM_plot}
\end{figure}



% -------------------------------------------------------------------
\subsubsection{Demodulador}

Considerando $\mathcal{E}_g = \int_{-\infty}^{\infty} |g(t)|^2 \,dt = 1$, a energia média da constelação pode ser calculada por $\epsilon$



%%
% Please add the following required packages to your document preamble:
