%--------------------------------------------------------------------------
% Exemplos
\section*{Exemplos}

% Place holder de texto
% \lipsum[1]

Para referenciar imagens~\ref{fig:brasao_UFC}, tabelas~\ref{tab:frequencia_tensao} e equações~\ref{eq:frequencia_ganho}.

\begin{figure}[!ht]
    \centering
    \includegraphics[width=0.15\textwidth]{adj/brasao.png}
    \caption{Exemplo de como adicionar uma imagem.}
    \label{fig:brasao_UFC}
\end{figure}

\begin{table}[!ht]
    \centering
    \begin{tabular}{|c|c|}
    \hline
    Frequência (Hz) & Tensão Máxima (V) \\ \hline
    0,558            & 12,11            \\ \hline
    2,132            & 11,15            \\ \hline
    4,822            & 8,62             \\ \hline
    \end{tabular}
    \caption{Frequência da onda de entrada e a tensão máxima da saída do circuito integrador.}
    \label{tab:frequencia_tensao}
\end{table}

\begin{equation}
    f_{gu} = A_{VD}\times f_{c}
    \label{eq:frequencia_ganho}
\end{equation}

% Exemplo de citação usando \cite{}
E quando tirar informação de alguma fonte, deve adicionar no formato de bibtex no arquivo refs.bib e por fim citá-los assim:~\cite{Sedra} \cite{Boylestad} \cite{Fonseca}, de modo que a seção de referência é criada e indexada diretamente com estes chamados da função.

\clearpage