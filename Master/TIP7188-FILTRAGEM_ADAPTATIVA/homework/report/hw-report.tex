\documentclass[a4paper, 12pt]{article}

\usepackage[utf8]{inputenc}
\usepackage[brazilian]{babel}
\usepackage{graphicx}
\usepackage{amssymb, mathrsfs, amsfonts, amsmath, esint, relsize, bm}
\usepackage{hyperref}
\usepackage{caption}
\usepackage{indentfirst} 
\usepackage{xcolor}
\usepackage{csquotes} % When using babel or polyglossia with biblatex, loading csquotes is recommended to ensure that quoted texts are typeset according to the rules of your main language.
\usepackage[ruled,vlined]{algorithm2e} % Algorithm
\usepackage{pdfpages} % Include full PDF file

% References
\usepackage[style=ieee]{biblatex}
\addbibresource{refs.bib}

% \usepackage{hhline}
% \usepackage{color}
% \usepackage{soul}
% \usepackage[table,xcdraw]{xcolor}
% \usepackage{enumerate}

%% My commands
% \newcommand{\myref}[1]{{\color{blue} \ref{#1}}} % Display a blue color for linking figures, tables, equations, etc..
% \newcommand{\mywidth}{.6} % Standard for figures
% \hypersetup{linkcolor=blue, filecolor=magenta, urlcolor=cyan}

% Change background color
% \usepackage{pagecolor,lipsum}% http://ctan.org/pkg/{pagecolor,lipsum}

% \definecolor{maroon}{cmyk}{0,0.87,0.68,0.32}

\begin{document}
% \pagecolor{yellow!50!orange} % ``marcador'' de página feita (movimentas ou comentá-lo de acordo com o avanço do trabalho)

%--------------------------------------------------------------------------
% Capa do relatório
\begin{titlepage}
    \begin{center}
        \includegraphics[width=2cm]{adj/brasao.png}\\
        {\large {Universidade Federal do Ceará}}\\
        {\large {Centro de Tecnologia}}\\
        {\large {Departamento de Engenharia de Teleinformática}}\\
        {\large {Sistemas de Comunicações Digitais - TI0069}}
    \end{center}

    \vspace{100pt}
    
    \begin{center}
        {\large \textbf {Trabalho 01: Modulação Digital}}
    \end{center}
    
    \vspace{100pt}
    
    \begin{table}[h]
    \begin{tabular}{ll}
        \textbf{Aluno:}         &       \\
        Lucas de Souza Abdalah  & 385472
    \end{tabular}
    \end{table}
    
    \begin{table}[h]
    \begin{tabular}{l}
        \textbf{Professor:} André Almeida   \\
        \textbf{Data de Entrega do Relatório:} 28/03/2021
    \end{tabular}
    \end{table}
    
    \vspace{\fill}
    
    \begin{center}
        Fortaleza\\
        2021
    \end{center}
    
    \end{titlepage}
    
    %--------------------------------------------------------------------------
    % Sumario do relatorio
    
    \tableofcontents
    \thispagestyle{empty}
    \clearpage

% ------------------------

\section{Lista 1: Estatísticas de Segunda Ordem}

\subsection{Média e Autocorrelação}
\todo[inline, color=yellow!30]{Organizar}


\subsection{Processos Escationários}
\todo[inline, color=yellow!30]{Organizar}


\subsection{Matriz de Autocorrelação}
\todo[inline, color=yellow!30]{Organizar}


\subsection{Matriz Definida Positiva}
\todo[inline, color=yellow!30]{Organizar}


\subsection{Covariância e correlação}
\todo[inline, color=yellow!30]{Organizar}


\subsection{Função de autocorrelação}
\todo[inline, color=yellow!30]{Organizar}


\includepdf[pages={-},addtotoc={1,subsection,1,Exercícios Propostos,p1}]{C:/Users/lucasabdalah-dell/Documents/GitHub/Courses-HWs/Master/TIP7188-FILTRAGEM_ADAPTATIVA/homework/report/listas/Lista_exercicios_1.pdf}
\clearpage
\section{Lista 2: Filtragem Linear Ótima}

\subsection{Filtragem Ótima}

\subsection{Erro Médio Quadrático Mínimo}

\subsection{Cancelamento de Ruído}

\subsection{Predição Ótima}

\subsection{Superfície de Erro}
\includepdf[pages={-},addtotoc={1,subsection,1,Exercícios Propostos,p1}]{C:/Users/lucasabdalah-dell/Documents/GitHub/Courses-HWs/Master/TIP7188-FILTRAGEM_ADAPTATIVA/homework/report/listas/Lista_exercicios_2.pdf}
\clearpage
\section{Lista 3: Algoritmos Recursivos}

\subsection{Algoritmo LMF}

\subsection{Algoritmo LMS}

\subsection{Algoritmo LMS Normalizado}

\subsection{Equalização de Canais}

\subsection{Identificação de Sistemas}

\subsection{Equalização Adaptativa}

\includepdf[pages={-},addtotoc={1,subsection,1,Exercícios Propostos,p1}]{C:/Users/lucasabdalah-dell/Documents/GitHub/Courses-HWs/Master/TIP7188-FILTRAGEM_ADAPTATIVA/homework/report/listas/Lista_exercicios_3.pdf}
\clearpage
\section{Lista 4: Método dos Mínimos Quadrados}

\subsection{Algoritmo RLS}
\todo[inline, color=yellow!30]{Organizar}


\subsection{Erro de Estimação a Priori}
\todo[inline, color=yellow!30]{Organizar}


\subsection{Preditor Adaptativo}
\todo[inline, color=yellow!30]{Organizar}


\subsection{Equalização de Canais}
\todo[inline, color=yellow!30]{Organizar}


\subsection{Equalização Adaptativa}
\todo[inline, color=yellow!30]{Organizar}


\includepdf[pages={-},addtotoc={1,subsection,1,Exercícios Propostos,p1}]{C:/Users/lucasabdalah-dell/Documents/GitHub/Courses-HWs/Master/TIP7188-FILTRAGEM_ADAPTATIVA/homework/report/listas/Lista_exercicios_4.pdf}
\clearpage

% Bibliografia
% LateX vai gerar as ``Referências'' automaticamente
% usando a função \cite{nome} do pacote BibTeX é possível
% "puxar todas as informações do arquivo 'refs.bib'
% O nome do arquivo é o primeiro parâmetro de cada referência
% Um exemplo esta é utilizado na primeira seção
% \AtNextBibliography{\small}             % To set a smaller font size for bibliography
% \printbibliography[heading=bibintoc]    % Print the references

\end{document}