\section{Lista 3: Algoritmos Recursivos} % <-----------------------------------------------------------------------------


\subsection{Algoritmo LMF} % <-----------------------------------------------------------------------------

O algoritmo é apresentado e debatido no trabalho \textit{The Least Mean Square Fourth (LMF) Algorithm
and Its Family} (Widrow, 1984), que foi utilizado de inspiração para solução desse problema.

Escrevendo a definição de erro em função do sinal desejado e da saída, obtemos:
\begin{align*}
    e(n) &= d(n) - y(n), \\
    e(n) &= d(n) - \mathbf{w}^{\text{T}}(n)\mathbf{x}(n),
\end{align*}

Utilizando o gradiente do erro elevado a quarta ordem, é possível obter a expressão de recursão através de derivação implícita:
\begin{align*}
    \nabla_{\mathbf{w}} \mathbb{E}\{e^{4}(n)\} &= \frac{\partial \mathbb{E}\{e^{4}(n)\}}{\partial \mathbf{w}} \\
    &= \mathbb{E} \left\{ \frac{\partial e^{4}(n)}{\partial \mathbf{w}}\right\} \\
    &= \mathbb{E}\left\{ \frac{\partial e^{4}(n)}{\partial e(n)} \cdot \frac{\partial e(n) }{\partial \mathbf{w}}\right\}
\end{align*}

Abrindo a expressão da derivação implícita, temos que:
\begin{align*}
    \nabla_{\mathbf{w}} \mathbb{E}\{e^{4}(n)\} &= \mathbb{E}\left\{4 e^{3}(n) \frac{\partial (d(n) - \mathbf{w}^{\text{T}}(n)\mathbf{x}(n)) }{\partial \mathbf{w}}\right\} \\
    &= \mathbb{E}\{4 e^{3}(n) (0 - \mathbf{x}(n))\} \\
    &= - 4 \mathbb{E}\{e^{3}(n) \mathbf{x}(n)\}
\end{align*}

Dado o resultado obtido em função de $\mathbb{E}\{e^{4}(n)\}$, visando minimizá-lo é necessário que $x(n)$ seja ortogonal à $e(n)$. Isto implica que:
\begin{align*}
    \mathbb{E}\{(d(n) - \mathbf{w}^{\text{T}}(n)\mathbf{x}(n))^{3} \mathbf{x}(n)\} = 0
\end{align*}

Ela permite demonstrar que a expressão converge em média, dado os seguintes parâmetros: $\sigma^{2}_{z}$ e $\lambda_{\text{max}}$, respectivamente, variância do ruído e o maior autovalor $\lambda_{\max}$ da matriz $\mathbf{R}_{x}$. Se o passo de aprendizado ($\mu_{\text{LMF}}$) for definido tal que:
\begin{align*}
    1 < \mu_{\text{LMF}} < \frac{1}{6 \sigma^{2}_{z} \lambda_{\text{max}}},
\end{align*}

O algoritmo recursivo do LMF é obtido a partir das expressões acima, reproduzindo a expressão semlhante ao do gradiente descendente, utilizando em problemas seguintes, de modo que:
\begin{align*}
    \mathbf{w}(n + 1) &= \mathbf{w}(n) - \mu_{\text{LMF}}\mathbf{g}_{w}(n) \\
    &= \mathbf{w}(n) + 4 \mu_{\text{LMF}} e^{3}(n) \mathbf{x}(n)
\end{align*}
\clearpage


\subsection{Algoritmo LMS} % <-----------------------------------------------------------------------------

\subsubsection*{Condição para convergência do algoritmo}

O erro nos coeficientes do filtro à cada iteração está associado à condição de convergência. Relacionando o erro dos coeficientes em um iteração qualquer com o filtro ótimo: $ \Delta \mathbf{w}(n) = \mathbf{w}(n) - \mathbf{w}_{\text{opt}} $. 

E isso permite aplicar a função de recurssão do LMS, de modo que:
\begin{align*}
    \Delta \mathbf{w}(n + 1) &= \Delta \mathbf{w}(n) + 2 \mu e(n) \mathbf{x}(n) \\
    &= \Delta \mathbf{w}(n) + 2 \mu \mathbf{x}(n) \left[e_{\text{opt}}(n) - \mathbf{x}^{\text{T}}(n) \Delta \mathbf{w}(n)\right] \\
    &= \left[ \mathbf{I} - 2 \mu \mathbf{x}(n) \mathbf{x}^{\text{T}}(n) \right] \Delta \mathbf{w}(n) + 2 \mu e_{\text{opt}}(n) \mathbf{x}(n)
\end{align*}

Aplicando o operador esperança em ambos os lados da expressão e reescrevendo o lado direito, partindo das relações $e(n) = e^{\text{T}}(n)$.
\begin{align*}
    \mathbb{E}\{\Delta \mathbf{w}(n + 1)\} &= \mathbb{E}\{\left[ \mathbf{I} - 2 \mu \mathbf{x}(n) \mathbf{x}^{\text{T}}(n) \right] \Delta \mathbf{w}(n) + 2 \mu e_{\text{opt}}(n) \mathbf{x}(n)\} \\
    &= \mathbb{E}\{\left[ \mathbf{I} - 2 \mu \mathbf{x}(n) \mathbf{x}^{\text{T}}(n) \right] \Delta \mathbf{w}(n)\} + 2 \mu \mathbb{E}\{e_{\text{opt}}(n) \mathbf{x}(n)\}
\end{align*}

Como em problemas anteriores, utilizamos novamente a vantagem do operador esperança ser um operador linear. Então, se temos simultaneamente que: $\mathbf{x}(n)$ é ortogonal a $e_{\text{opt}}(n)$ e $\Delta \mathbf{w}(n)$, logo:
\begin{align*}
    \mathbb{E}\{\Delta \mathbf{w}(n + 1)\} &= \left[ \mathbf{I} - 2 \mu \mathbb{E}\{\mathbf{x}(n) \mathbf{x}^{\text{T}}(n)\} \right] \mathbb{E}\{\Delta \mathbf{w}(n)\} \\
    &= \left( \mathbf{I} - 2 \mu \mathbf{R}_{x} \right) \mathbb{E}\{\Delta \mathbf{w}(n)\}
\end{align*}

Visando obter uma matriz diagonal para facilitar a análise de condicionamento do algoritmo, assume-se a existência uma matriz unitária que diagonaliza $\mathbf{R}_{x}$, $\mathbf{Q}$, onde:
\begin{align*}
    \mathbb{E}\{ \mathbf{Q}^{\text{T}} \Delta \mathbf{w}(n + 1) \} &= \left( \mathbf{I} - 2 \mu \mathbf{Q}^{\text{T}} \mathbf{R}_{x} \right) \mathbf{I} \mathbb{E}\{ \Delta \mathbf{w}(n)\} \\
    &= \left( \mathbf{I} - 2 \mu \mathbf{Q}^{\text{T}} \mathbf{R}_{x} \right) \mathbf{Q} \mathbf{Q}^{\text{T}} \mathbb{E}\{ \Delta \mathbf{w}(n)\} \\
    &= \left( \mathbf{I} - 2 \mu \mathbf{Q}^{\text{T}} \mathbf{R}_{x} \mathbf{Q} \right) \mathbb{E}\{ \mathbf{Q}^{\text{T}} \Delta \mathbf{w}(n)\} \\
\end{align*}

Reorganizando, tem-se que:
\begin{align*}
    \mathbb{E}\{\Delta \mathbf{w}'(n + 1)\} &= \left( \mathbf{I} - 2 \mu \mathbf{\Lambda} \right) \mathbb{E}\{\Delta \mathbf{w}'(n)\}
\end{align*}

A expressão anterior, pode ser expandida à esquerda para a análise de convergência do filtro, de modo que:
\begin{align*}
    \mathbb{E}\{ \Delta \mathbf{w}'(n + 1) \} &= \left( \mathbf{I} - 2 \mu \mathbf{\Lambda} \right)^{n + 1} \mathbb{E}\{\Delta \mathbf{w}'(0)\} \\
    &= \text{diag} \left[ (1 - 2 \mu \lambda_{1})^{n + 1}, (1 - 2 \mu \lambda_{2})^{n + 1}, \dots, (1 - 2 \mu \lambda_{N})^{n + 1}) \right] \mathbb{E}\{ \Delta \mathbf{w}'(0)\}
\end{align*}

Onde $\text{diag}(\cdot)$ é uma matriz diagonal contendo cada autovalor da matriz de autocorrelação, $\lambda_{n} \forall n \in \{1, \cdots, N\}$. Isso permite avaliar a condição de estabilidade apenas com propriedades relacionadas à $\lambda$. 

Finalmente, de acordo com a expressão anterior, temos que para garantir estabilidade na convergência, é necessário que o passo de aprendizado do algoritmo $\mu$ esteja contido entre $0$ e o maior autovalor. Desta forma é possível garantir que a medida que as dimensões dessa matriz aumenta, os valores vão reduzir, tendendo a zero:
\begin{align*}
    0 < \mu < \frac{1}{\lambda}_{\text{max}}
\end{align*}

\clearpage

\subsubsection*{Erro em excesso em média quadrática}
\todo[inline, color=yellow!30]{Organizar}

% O erro em excesso é normalmente ocasionado pelos termos ruidosos presentes no gradiente, impedindo que os coeficientes convirjam de forma exata para a solução ótima. 
% Podemos iniciar essa análise escrevendo a equação para o erro de estimação para um determinado instante $n$ como se segue

% \begin{align*}
%     e(n) &= d(n) - \mathbf{w}^{\text{T}}_{\text{opt}} \mathbf{x}(n) - \Delta \mathbf{w}^{\text{T}}(n) \mathbf{x}(n), \\
%     e(n) &= e_{\text{opt}}(n) - \Delta \mathbf{w}^{\text{T}}(n) \mathbf{x}(n),
% \end{align*}

% onde o erro quadrático é expresso por

% \begin{align*}
%     e^{2}(n) &= e^{2}_{\text{opt}}(n) - 2 e_{\text{opt}}(n) \Delta \mathbf{w}^{\text{T}}(n) \mathbf{x}(n) + \Delta \mathbf{w}^{\text{T}}(n) \mathbf{x}(n) \mathbf{x}^{\text{T}}(n) \Delta \mathbf{w}(n) ,
% \end{align*}

% se definirmos o erro quadrático médio como $\xi(n) = \mathbb{E}\{e^{2}(n)\}$ e o erro ótimo, leia-se mínimo, como $\xi_{\text{min}} = \mathbb{E}\{e^{2}_{\text{opt}}(n)\}$
% então podemos escrever

% \begin{align*}
%     \xi(n) = \xi_{\text{min}} - 2 \mathbb{E}\{e_{\text{opt}}(n) \Delta \mathbf{w}^{\text{T}}(n) \mathbf{x}(n)\} + \mathbb{E}\{\Delta \mathbf{w}^{\text{T}}(n) \mathbf{x}(n) \mathbf{x}^{\text{T}}(n) \Delta \mathbf{w}(n)\},
% \end{align*}

% considerando mais uma vez que $\mathbf{x}$ é ortogonal ao $\Delta \mathbf{w}^{\text{T}}(n)$ e ao $e_{\text{opt}}(n)$, simultaneamente, podemos simplificar a expressão ainda mais 

% \begin{align*}
%     \xi(n) = \xi_{\text{min}} - 2 \mathbb{E}\{\Delta \mathbf{w}^{\text{T}}(n)\} \mathbb{E}\{e_{\text{opt}}(n) \mathbf{x}(n)\} + \mathbb{E}\{\Delta \mathbf{w}^{\text{T}}(n) \mathbf{x}(n) \mathbf{x}^{\text{T}}(n) \Delta \mathbf{w}(n)\},
% \end{align*}

% onde podemos utilizar a propriedade $\mathbb{E}\{\Delta \mathbf{w}^{\text{T}}(n) \mathbf{x}(n) \mathbf{x}^{\text{T}}(n) \Delta \mathbf{w}(n)\} = \text{tr}(\mathbb{E}\{\Delta \mathbf{w}^{\text{T}}(n) \mathbf{x}(n) \mathbf{x}^{\text{T}}(n) \Delta \mathbf{w}(n)\})$ uma vez
% que sabemos que o traço de um escalar é o próprio escalar. Sendo assim, utilizando a propriedade cíclica do operador traço escrevemos

% \begin{align*}
%     \xi(n) &= \xi_{\text{min}} - 2 \mathbb{E}\{\Delta \mathbf{w}^{\text{T}}(n)\} \mathbb{E}\{e_{\text{opt}}(n) \mathbf{x}(n)\} + \text{tr}(\mathbb{E}\{\Delta \mathbf{w}^{\text{T}}(n) \mathbf{x}(n) \mathbf{x}^{\text{T}}(n) \Delta \mathbf{w}(n)\}), \\
%     \xi(n) &= \xi_{\text{min}} - 2 \mathbb{E}\{\Delta \mathbf{w}^{\text{T}}(n)\} \mathbb{E}\{e_{\text{opt}}(n)\mathbf{x}(n)\} + \mathbb{E}\{\text{tr}[\Delta \mathbf{w}^{\text{T}}(n) \mathbf{x}(n) \mathbf{x}^{\text{T}}(n) \Delta \mathbf{w}(n)]\}, \\
%     \xi(n) &= \xi_{\text{min}} - 2 \mathbb{E}\{\Delta \mathbf{w}^{\text{T}}(n)\} \mathbb{E}\{e_{\text{opt}}(n)\mathbf{x}(n)\} + \mathbb{E}\{\text{tr}[\mathbf{x}(n) \mathbf{x}^{\text{T}}(n) \Delta \mathbf{w}(n) \Delta \mathbf{w}^{\text{T}}(n)]\}, \\
%     \xi(n) &= \xi_{\text{min}} + \mathbb{E}\{\text{tr}[\mathbf{x}(n) \mathbf{x}^{\text{T}}(n) \Delta \mathbf{w}(n) \Delta \mathbf{w}^{\text{T}}(n)]\}, \\
%     \xi(n) &= \xi_{\text{min}} + \text{tr}(\mathbb{E}\{\mathbf{x}(n) \mathbf{x}^{\text{T}}(n) \Delta \mathbf{w}(n) \Delta \mathbf{w}^{\text{T}}(n)\}), \\
%     \xi(n) &= \xi_{\text{min}} + \text{tr}(\mathbb{E}\{\mathbf{x}(n) \mathbf{x}^{\text{T}}(n)\} \mathbb{E}\{\Delta \mathbf{w}(n) \Delta \mathbf{w}^{\text{T}}(n)\}), \\
%     \xi(n) &= \xi_{\text{min}} + \text{tr}(\mathbf{R}_{x} \mathbb{E}\{\Delta \mathbf{w}(n) \Delta \mathbf{w}^{\text{T}}(n)\}), \\
%     \xi(n) &= \xi_{\text{min}} + \text{tr}(\mathbb{E}\{\mathbf{R}_{x} \Delta \mathbf{w}(n) \Delta \mathbf{w}^{\text{T}}(n)\}), 
% \end{align*}

% onde o processo justifica-se pela intercambialidade entre os operadores traço e valor esperador e pela ortogonalidade mencionada anteriormente. Em sequência podemos definir o erro em excesso por

% \begin{align}
%     \xi(n) - \xi_{\text{min}} &= \text{tr}(\mathbb{E}\{\mathbf{R}_{x} \Delta \mathbf{w}(n) \Delta \mathbf{w}^{\text{T}}(n)\}), \\
%     \Delta \xi(n) &= \text{tr}(\mathbb{E}\{\mathbf{R}_{x} \Delta \mathbf{w}(n) \Delta \mathbf{w}^{\text{T}}(n)\}), \label{excesso1}
% \end{align}

% e adicionalmente, pela definição de uma transformação de similaridade onde garantimos que exista $\mathbf{Q} \mathbf{Q}^{\text{T}} = \mathbf{I}$ capaz de diagonalizar a matriz de autocorrelação, podemos reescrever a Equação (\ref{excesso1}) da seguinte forma

% \begin{align}
%     \Delta \xi(n) &= \text{tr}(\mathbb{E}\{\mathbf{Q} \mathbf{Q}^{\text{T}} \mathbf{R}_{x} \mathbf{Q} \mathbf{Q}^{\text{T}} \Delta \mathbf{w}(n) \Delta \mathbf{w}^{\text{T}}(n) \mathbf{Q} \mathbf{Q}^{\text{T}}\}), \\
%     \Delta \xi(n) &= \text{tr}(\mathbb{E}\{\mathbf{Q} \mathbf{\Lambda} \mathbf{Q}^{\text{T}} \Delta \mathbf{w}(n) \Delta \mathbf{w}^{\text{T}}(n) \mathbf{Q} \mathbf{Q}^{\text{T}}\}), \\
%     \Delta \xi(n) &= \text{tr}(\mathbb{E}\{\mathbf{Q} \mathbf{\Lambda} \text{cov} \left[\Delta \mathbf{w}'(n)\right] \mathbf{Q}^{\text{T}}\}), \\
%     \Delta \xi(n) &= \text{tr}(\mathbb{E}\{\mathbf{\Lambda} \text{cov} \left[\Delta \mathbf{w}'(n)\right] \mathbf{Q}^{\text{T}} \mathbf{Q}\}), \\
%     \Delta \xi(n) &= \text{tr}(\mathbb{E}\{\mathbf{\Lambda} \text{cov} \left[\Delta \mathbf{w}'(n)\right]\}), \label{excesso2}
% \end{align}

% onde é sabido que $\text{cov} \left[\Delta \mathbf{w}(n)\right] = \mathbb{E}\{(\mathbf{w}(n) - \mathbf{w}_{\text{opt}}) (\mathbf{w}(n) - \mathbf{w}_{\text{opt}})^{\text{T}}\}$ e $\Delta \mathbf{w}'(n) = \mathbf{Q}^{\text{T}} \Delta \mathbf{w}(n) \Delta \mathbf{w}^{\text{T}}(n) \mathbf{Q}$.
% Enfim, a partir das orientações presentes no livro texto da disciplina a respeito da matriz de covariância do vetor de erros é possível ainda escrever a Equação (\ref{excesso2}) como

% \begin{align*}
%     \Delta \xi(n) &= \sum^{N}_{i = 1} \lambda_{i} v_{i}'(n) = \mathbf{\lambda}^{\text{T}} \mathbf{v}'(n),
% \end{align*}

% onde $\mathbf{\lambda}$ é um vetor contendo todos os autovalores da matriz de autocorrelação e $\mathbf{v}(n)$ é um vetor que contém os elementos da diagonal de $\text{cov} \left[\Delta \mathbf{w}'(n)\right]$. Em suma, podemos 
% expressar o $i$th elemento do vetor diagonal $\mathbf{v}'(n + 1)$, seguindo orientações disponíveis no livro texto da disciplina, com o seguinte equacionamento

% \begin{align*}
%     v_{i}'(n + 1) &= (1 - 4 \mu \lambda_{i} + 8 \mu^{2} \lambda^{2}_{i}) v_{i}'(n) + 4 \mu^{2} \lambda_{i} \sum^{N}_{j = 0} \lambda_{j} v_{j}'(n) + 4 \mu^{2} \sigma^{2}_{z} \lambda_{i}.
% \end{align*}

% Entretanto, se considerarmos que $v_{i}'(n + 1) \approx v_{i}'(n)$ quando $n \rightarrow \infty$ podemos simplificar a expressão, além de que 
% também podemos realizar uma operação de soma total dos parâmetros para obter o erro em excesso total

% \begin{align*}
%     \notag &\sum^{N}_{i = 1} v_{i}'(n) = \sum^{N}_{i = 1} (1 - 4 \mu \lambda_{i} + 8 \mu^{2} \lambda^{2}_{i}) v_{i}'(n) + 4 \mu^{2} \sum^{N}_{i = 1} \lambda_{i} \sum^{N}_{j = 0} \lambda_{j} v_{j}'(n) + 4 \mu^{2} \sigma^{2}_{z} \sum^{N}_{i = 1} \lambda_{i}, \\
%     \notag &\sum^{N}_{i = 1} v_{i}'(n) = \sum^{N}_{i = 1} v_{i}'(n) - 4 \mu \sum^{N}_{i = 1} \lambda_{i} v_{i}'(n)  + 8 \mu^{2} \sum^{N}_{i = 1} \lambda^{2}_{i} v_{i}'(n) + 4 \mu^{2} \sum^{N}_{i = 1} \lambda_{i} \sum^{N}_{j = 0} \lambda_{j} v_{j}'(n) + 4 \mu^{2} \sigma^{2}_{z} \sum^{N}_{i = 1} \lambda_{i}, \\
%     \notag &\sum^{N}_{i = 1} v_{i}'(n) - \sum^{N}_{i = 1} v_{i}'(n) + 4 \mu \sum^{N}_{i = 1} \lambda_{i} v_{i}'(n) - 4 \mu^{2} \sum^{N}_{i = 1} \lambda_{i} \sum^{N}_{j = 0} \lambda_{j} v_{j}'(n) = 4 \mu^{2} \sigma^{2}_{z} \sum^{N}_{i = 1} \lambda_{i} + 8 \mu^{2} \sum^{N}_{i = 1} \lambda^{2}_{i} v_{i}'(n), \\
%     \notag &4 \mu \sum^{N}_{j = 1} \lambda_{j} v_{j}'(n)(1 - \mu \sum^{N}_{i = 1} \lambda_{i}) = 4 \mu (\mu \sigma^{2}_{z} \sum^{N}_{i = 1} \lambda_{i} + 2 \mu \sum^{N}_{i = 1} \lambda^{2}_{i} v_{i}'(n)), \\
%     \notag &\sum^{N}_{j = 1} \lambda_{j} v_{j}'(n) = \frac{\mu \sigma^{2}_{z} \sum^{N}_{i = 1} \lambda_{i} + 2 \mu \sum^{N}_{i = 1} \lambda^{2}_{i} v_{i}'(n)}{1 - \mu \sum^{N}_{i = 1} \lambda_{i}}, \\
%     \notag &\sum^{N}_{j = 1} \lambda_{j} v_{j}'(n) \approx \frac{\mu \sigma^{2}_{z} \sum^{N}_{i = 1} \lambda_{i} }{1 - \mu \sum^{N}_{i = 1} \lambda_{i}}, \\
%     &\sum^{N}_{j = 1} \lambda_{j} v_{j}'(n) = \frac{\mu \sigma^{2}_{z} \text{tr}(\mathbf{R}_{x})}{1 - \mu \text{tr}(\mathbf{R}_{x})},
% \end{align*}

% onde foi considerado que o termo $2 \mu \sum^{N}_{i = 1} \lambda^{2}_{i} v_{i}'(n)$ apresenta contribuição insignificante para o valor absoluto do numerador. Entretanto, é mencionado no livro texto que tal aproximação é de prova complexa, mas que normalmente 
% se verifica verdadeira para pequenos valores do passo de aprendizado $\mu$. Portanto, o erro em excesso pode ser prontamente descrito pela expressão que se segue

% \begin{align*}
%     \xi_{\text{excesso}} &= \underset{n \rightarrow \infty}{\text{lim}} \Delta \xi(n) \approx \frac{\mu \sigma^{2}_{z} \text{tr}(\mathbf{R}_{x})}{1 - \mu \text{tr}(\mathbf{R}_{x})}.
% \end{align*}

% Vale ainda expor que podemos considerar $1 - \mu \text{tr}(\mathbf{R}_{x}) \approx 1$ para valores muito pequenos de $\mu$, obtendo assim uma versão aproximada para o erro em excesso dada por

% \begin{align*}
%     \xi_{\text{excesso}} &= \underset{n \rightarrow \infty}{\text{lim}} \Delta \xi(n) \approx \mu \sigma^{2}_{z} \text{tr}(\mathbf{R}_{x}).
% \end{align*}

\clearpage


\subsection{Algoritmo LMS Normalizado} % <-----------------------------------------------------------------------------
A expressão recursiva do algoritmo NLMS para atualização dos coeficientes de filtro é dada por:
\begin{align*}
    \mathbf{w}(k+1) = \mathbf{w}(k) + \frac{\mu_{norm}}{\gamma + \mathbf{x}^{\text{T}}(k) \mathbf{x}(k)} \mathbf{e}(k) \mathbf{x}(k),
\end{align*}

A partir da relação com o algortimo LMS, asssumindo um valor médio de $\mu$, na direção de $2 \mathbf{e}(k) \mathbf{x}(k)$ está diretamente relacionado com a expressão: 
\begin{align*}
    \frac{\mu_{norm}}{2 \text{trace}(\mathbf{R_{xx}})}
\end{align*}

Finalmente, isso permite limitar à direita o valor de convergência do NLMS, tendo como parâmetro as equações utilizadas na atualização do LMS.
\begin{align*}
    0 &< \frac{\mu_{norm}}{2 \text{trace}(\mathbf{R_{xx}})} < \frac{1}{\text{trace}(\mathbf{R_{xx}})} \\
    0 &< \mu_{norm} < 2, 
\end{align*}
        

\subsection{Equalização de Canais} % <-----------------------------------------------------------------------------

\subsubsection*{Equalizado Ótimo e plano Z}

Considerando o sinal gaussiano branco $x(n)$, a saída do canal $y(n)$ é dada por: $ y(n) = x(n) + 1.6 x(n - 1)$. Consequentemente, tem a seguinte a matriz de autocorrelação:
\begin{align*}
    \mathbf{R}_{y} =
    \begin{bmatrix}
        \mathbb{E}\{y(n)y^{\text{H}}(n)\} & \mathbb{E}\{y(n)y^{\text{H}}(n - 1)\} \\
        \mathbb{E}\{y(n - 1)y^{\text{H}}(n)\} & \mathbb{E}\{y(n - 1)y^{\text{H}}(n - 1)\}
    \end{bmatrix},
\end{align*}

Aplicando os valores em cada expressão de correlação, e sabendo que sinal tem média nula e suas amostras são independentes, temos que matriz de autocorrelação teórica:
\begin{align*}
    \mathbf{R}_{y} =
    \begin{bmatrix}
        3.56 & 1.60 \\
        1.60 & 3.56
    \end{bmatrix}
\end{align*}

O vetor de correlação cruzada é dado por:
\begin{align*}
    \mathbf{p}_{yd} =
    \begin{bmatrix}
        \mathbb{E}\{y(n)d(n)\} \\
        \mathbb{E}\{y(n - 1)d(n)\}
    \end{bmatrix} = 
    \begin{bmatrix}
        1 \\
        0
    \end{bmatrix},
\end{align*}
buscando obter a maior correlação possivel com o sinal desejado, sem que haja dependência de amostras em instantes diferentes. Por fim, temos que os coeficientes ótimos (Wiener) do equalizador são:
\begin{align*}
    \mathbf{w}_{\text{opt}} &= \mathbf{R}^{-1}_{y} \mathbf{p}_{yd} \\
    &= \begin{bmatrix}
        0.35 &  -0.16 \\
        -0.16 & 0.35
    \end{bmatrix} 
    \begin{bmatrix}
        1 \\
        0
    \end{bmatrix} \\
    &= \begin{bmatrix}
        0.35 \\
        -0.16
    \end{bmatrix} 
\end{align*}

\begin{figure}[!htp]
    \centering
    \includegraphics[width=0.8\textwidth]{fig/hw3p4-zeros.pdf}
    \caption{Zeros do canal e do equalizador no plano-z.}
    \label{fig:hw3p4-zeros}
\end{figure}

Levando em consideração todas as informações obtidas, é possível calculcar o diagram de zeros do filtro e do canal na Fig.~\ref{fig:hw3p4-zeros}.

\subsubsection*{Filtro de Erro de Predição Direta de Passo Unitário}
O filtro de predição direta de passo unitário é dado por:
\begin{align*}
    \hat{x}(n) &= \sum^{M + k - 1}_{i = k} w_{f,i} x(n - i) \\
    &= \sum^{2}_{i = 1} w_{f,i} x(n - i) \\
    &= \mathbf{w}^{\text{T}}_{f} \mathbf{x}(n - 1),
\end{align*}

Tendo que o erro quadrático médio (MSE) é:
\begin{align*}
    \mathbb{E}\{e^{2}(n)\} &= \mathbb{E}\{(x(n) - \hat{x}(n) )^{2}\} \\
    &= \mathbf{r}_{x}(0) - 2 \mathbf{w}^{\text{T}}_{f} \mathbf{r}_{x,f} + \mathbf{w}^{\text{T}}_{f} \mathbf{R}_{x} \mathbf{w}_{f}
\end{align*}
E a solução ótima é: $ \mathbf{w}_{f,\text{opt}} = \mathbf{R}^{-1}_{x} \mathbf{r}_{x,f} $.

Teremos assim que a mesma matriz de autocorrelação e o vetor de correlação cruzada serão definidos por
\begin{align*}
    \mathbf{r}_{y,f} &= 
    \begin{bmatrix}
        r_{y}(1) \\
        r_{y}(2)
    \end{bmatrix} \\
    & =
    \begin{bmatrix}
        \mathbb{E}\{y(n) y(n - 1)\} \\
        \mathbb{E}\{y(n) y(n - 2)\}
    \end{bmatrix} \\
    & = 
    \begin{bmatrix}
        1.60 \\
        0
    \end{bmatrix}.
\end{align*}

Logo, os coeficientes do filtro são:
\begin{align*}
    \mathbf{w}_{f,\text{opt}} &= 
    \begin{bmatrix}
        0.35 & -0.16 \\
        -0.16 & 0.35
    \end{bmatrix}
    \begin{bmatrix}
        1.60 \\
        0
    \end{bmatrix} \\ 
    &=
    \begin{bmatrix}
        0.56 \\
        -0.26
    \end{bmatrix}.
\end{align*}

Por fim, podemos verificar o ponto de zero do filtro, como antecipado no Fig.~\ref{fig:hw3p4-zeros}, $ W(z) = 0.56 - 0.26 z^{-1}$:
\begin{align*}
    0.56 - 0.26 z^{-1} &= 0 \\
    z &= 0.45
\end{align*}

\subsubsection*{Curvas de MSE e de Nível dos algoritmos}

Como apresentado anteriormente, na seção 2, questão 5, o conceito de superfície de erro utilizado para traçar as curvas de nível e MSE é baseado respectivamente nas funções $J(w) = \sigma^{2}_{d} - 2\mathbf{w}^{\top}\mathbf{p}_{\mathbf{X} d} + w^{\top}\mathbf{R}_{X}\mathbf{w} $ e $J(w) = \mathbb{E}\{e^{2}(n)\}$.

\begin{figure}[!htp]
    \centering
    \includegraphics[width=0.80\textwidth]{fig/hw3p4-dga.pdf}
    \caption{Resultados da implementação do algoritmo gradiente determinístico com $N = 1000$ amostras, filtro de ordem $M = 2$ e parâmtro $\mu = 10^{-2}$. \textbf{Superior:} Evoulação da curva MSE. \textbf{Inferior:} Caminho percorrido até o ponto de convergência, i.e, filtro de Wiener.}
    \label{fig:hw3p4-dga}
\end{figure}

\begin{figure}[!htp]
    \centering
    \includegraphics[width=0.80\textwidth]{fig/hw3p4-newton.pdf}
    \caption{Resultados da implementação do algoritmo Newton com $N = 1000$ amostras, filtro de ordem $M = 2$ e parâmtro $\mu = 0.5 \times 10^{-2}$. \textbf{Superior:} Evoulação da curva MSE. \textbf{Inferior:} Caminho percorrido até o ponto de convergência, i.e, filtro de Wiener.}
    \label{fig:hw3p4-newton}
\end{figure}

\begin{figure}[!htp]
    \centering
    \includegraphics[width=0.80\textwidth]{fig/hw3p4-lms.pdf}
    \caption{Resultados da implementação do algoritmo LMS com $N = 1000$ amostras, filtro de ordem $M = 2$ e parâmtro $\mu = 10^{-3}$. \textbf{Superior:} Evoulação da curva MSE. \textbf{Inferior:} Caminho percorrido até o ponto de convergência, i.e, filtro de Wiener.}
    \label{fig:hw3p4-lms}
\end{figure}

\begin{figure}[!htp]
    \centering
    \includegraphics[width=0.80\textwidth]{fig/hw3p4-nlms.pdf}
    \caption{Resultados da implementação do algoritmo NLMS com $N = 1000$ amostras, filtro de ordem $M = 2$ e parâmtros $\mu = 0.5 \times 10^{-1}$ and $\gamma = 0.5 $. \textbf{Superior:} Evoulação da curva MSE. \textbf{Inferior:} Caminho percorrido até o ponto de convergência, i.e, filtro de Wiener.}
    \label{fig:hw3p4-nlms}
\end{figure}


As Figuras \ref{fig:hw3p4-dga}, \ref{fig:hw3p4-newton}, \ref{fig:hw3p4-lms} e \ref{fig:hw3p4-nlms} apresentam o comportamento do erro quadrático médio (MSE) e as curvas de convergência sobre a superfície MSE para cada algoritmo implementado. A ordem de $M = 2$ foi utilizada para todos os filtros, consequentemente o vetor de pesos na atualização possui apenas 2 coeficientes atualizados por iteração. O desempenho médio é semelhante, com certa vantagem para o NLMS.

Para os métodos determinísticos, é possível observar que há convergência organizada e suave. Isso é esperado, dado que o algoritmo utiliza o conhecimento dos coeficientes ideais do filtro. Enquanto para os métodos estocásticos, é visível algumas regiões de desordem na convergência, isso é dado como consequência da utilização das aproximações estatísticas instantâneas do sinal para o cálculo dos coeficientes. Ao comparar o LMS e sua versão normalizada, o primeiro apresenta uma maior estabilidade de convergência, enquanto o segundo apresentar uma numvem de pontos bem menos densa em torno da solução de Wiener, além de pontos que aparentemente se afastam da solução, consequência das iterações iniciais.

\subsubsection*{Número de condicionamento}
O número de condicionamento consiste no quociente entre o maior $\lambda_{\max}$ e menor $\lambda_{\min}$ autovalor da matrix $\mathbf{R}_{x}$.

A partir da equação $\lambda^{2} - 7.12 \lambda + 10.11 = 0$, obtém-se as raízes do polinômio e consequentemente os autovalores e autovetores da matriz.

Utilizando o MATLAB para calcular o quociente entre autovalores máximo e mínimo, obtemos que:
\begin{align*}
    \mathbb{C} (\mathbf{R}_{x}) = \frac{5.160}{1.960} = 2.633
\end{align*}

\subsubsection*{Modelo de canal para número de condicionamento menor/maior que 5}

Definindo uma função de transferência do canal: $H(z) = a_{0} + a_{1}z^{-1}$, pode-se obter a matriz de autocorrelação, visando obter o polinômio característico: $\lambda^{2} + b \lambda + c = 0$.
\begin{align*}
    \mathbf{R}_{y} =
    \begin{bmatrix}
        a_{0} + a^{2}_{1} & a_{1}\\
        a_{1} & a_{0} + a^{2}_{1}
    \end{bmatrix},
\end{align*}

Reorganizando a equação característica obtida a partir da matriz, tem-se que:
\begin{align*}
    &\lambda^{2} \overbrace{- 2 (a_{0} + a^{2}_{1})}^{b} \lambda + \overbrace{(a_{0} + a^{2}_{1})^{2} - a^{2}_{1}}^{c} = 0 \\
    &\lambda^{2} + b \lambda + c = 0
\end{align*}

A solução de equação de 2º grau pode ser obtida através equação de Bháskara, de modo que:

\begin{align*}
    \mathbb{C} (\mathbf{R}_{x}) &= \frac{\lambda_{\text{max}}}{\lambda_{\text{min}}} \\
     &= \frac{2 (a_{0} + a^{2}_{1}) + \sqrt{4 (a_{0} + a^{2}_{1})^{2} - 4 (a_{0} + a^{2}_{1})^{2} + 4 a^{2}_{1}}}{2 (a_{0} + a^{2}_{1}) - \sqrt{4 (a_{0} + a^{2}_{1})^{2} - 4 (a_{0} + a^{2}_{1})^{2} + 4 a^{2}_{1}}} \\
     &= \frac{a_{0} + a^{2}_{1} + a_{1}}{a_{0} + a^{2}_{1} - a_{1}}
\end{align*}

As relações acima permitem estabelecer as inequações que limitam o número de condicionamento: $ 5 (a_{0} + a^{2}_{1} - a_{1}) \leq a_{0} + a^{2}_{1} + a_{1} \leq 5 (a_{0} + a^{2}_{1} - a_{1})$.

% \clearpage


\subsection{Identificação de Sistemas} % <-----------------------------------------------------------------------------

\subsubsection*{Limite Superior para $\mu$}

Para definir o limite superior do fator de aprendizado, i.e, assegurar estabilidade, é necessário obter $\lambda_{\max}$ de $R_X$. 

Para isso é necessário desenvolver as equações que caracterizam o sistema, função de transferência e a representação da saída no tempo, respectivamente:
\begin{align*}
    H(z) &= \frac{1 - z^{-12}}{1 - z^{-1}} \\
    &= \frac{(1 - z^{-1})(\sum_{k= 0}^{11} z^{-k})}{1 - z^{-1}} \\
    &= \sum_{k = 0}^{11} z^{-k}
\end{align*}
\begin{align*}
    y(n) = \sum_{k = 0}^{11} x(n-k)
\end{align*}

Tendo essas informações, é possível estimar a $R_X$ utilizando o MATLAB, mostrada na Tabela ~\ref{tab:rxx}.
\begin{table}[!htb]
    \centering
    \begin{tabular}{llllllllllll}
    11.9509 & 10.9518 & 9.9530  & 8.9546  & 7.9567  & 6.9589  & $\dots$ & 0.9785  \\
    10.9518 & 11.9509 & 10.9518 & 9.9530  & 8.9546  & 7.9567  & $\dots$ & 1.9749  \\
    9.9530  & 10.9518 & 11.9509 & 10.9518 & 9.9530  & 8.9546  & $\dots$ & 2.9716  \\
    8.9546  & 9.9530  & 10.9518 & 11.9509 & 10.9518 & 9.9530  & $\dots$ & 3.9684  \\
    7.9567  & 8.9546  & 9.9530  & 10.9518 & 11.9509 & 10.9518 & $\dots$ & 4.9654  \\
    6.9589  & 7.9567  & 8.9546  & 9.9530  & 10.9518 & 11.9509 & $\dots$ & 5.9620  \\
    5.9620  & 6.9589  & 7.9567  & 8.9546  & 9.9530  & 10.9518 & $\dots$ & 6.9589  \\
    4.9654  & 5.9620  & 6.9589  & 7.9567  & 8.9546  & 9.9530  & $\dots$ & 7.9567  \\
    3.9684  & 4.9654  & 5.9620  & 6.9589  & 7.9567  & 8.9546  & $\dots$ & 8.9546  \\
    2.9716  & 3.9684  & 4.9654  & 5.9620  & 6.9589  & 7.9567  & $\dots$ & 9.9530  \\
    1.9749  & 2.9716  & 3.9684  & 4.9654  & 5.9620  & 6.9589  & $\dots$ & 10.9518 \\
    0.9785  & 1.9749  & 2.9716  & 3.9684  & 4.9654  & 5.9620  & $\dots$ & 11.9509
    \end{tabular}
    \caption{Matriz de autocorrelação $R_X$, com simulação de Monte Carlo, utilizando 10000 iterações.}
    \label{tab:rxx}
\end{table}

De acordo com o autovalor máximo, obtido no MATLAB, temos o seguinte intervalo de convergência:
\begin{align*}
    0 < \mu < \frac{1}{97}
\end{align*}

\subsubsection*{O algoritmo para $\frac{\mu_{\text{max}}}{2}$, $\frac{\mu_{\text{max}}}{10}$ e $\frac{\mu_{\text{max}}}{50}$} 


As Figuras~\ref{fig:mu_2}, \ref{fig:mu_10} e \ref{fig:mu_50} mostram o desempenho de cada filtro, de acordo com os parâmetros, e o apresentam o comportamento dos coeficientes na convergência do algoritmo. 

A primeira vista, o rastreio do resposta em frequência é o que deixa mais evidente a relação entre a diminuição do $\mu$ e a piora do desempenho. Isso se dá, pois ao passo que o $\mu$ reduz-se, a flexibilidade de adaptação diminui concomitantemente, tornando mais difícil do filtro acompanhar mudanças no canal.

\begin{figure}[!htp]
    \centering
    \includegraphics[width=0.9\textwidth]{C:/Users/lucasabdalah-dell/Documents/GitHub/Courses-HWs/Master/TIP7188-FILTRAGEM_ADAPTATIVA/homework/code/figures/hw3p5b-mu02.pdf}
    \caption{$\text{Amostras} = 5000$, $M = 15$, $\mu = \frac{\mu_{\text{max}}}{2}$}
    \label{fig:mu_2}
\end{figure}

\begin{figure}[!htp]
    \centering
    \includegraphics[width=0.9\textwidth]{C:/Users/lucasabdalah-dell/Documents/GitHub/Courses-HWs/Master/TIP7188-FILTRAGEM_ADAPTATIVA/homework/code/figures/hw3p5b-mu10.pdf}
    \caption{$\text{Amostras} = 5000$, $M = 15$, $\mu = \frac{\mu_{\text{max}}}{10}$}
    \label{fig:mu_10}
\end{figure}

\begin{figure}[!htp]
    \centering
    \includegraphics[width=0.9\textwidth]{C:/Users/lucasabdalah-dell/Documents/GitHub/Courses-HWs/Master/TIP7188-FILTRAGEM_ADAPTATIVA/homework/code/figures/hw3p5b-mu50.pdf}
    \caption{$\text{Amostras} = 5000$, $M = 15$, $\mu = \frac{\mu_{\text{max}}}{50}$}
    \label{fig:mu_50}
\end{figure}

\clearpage

\subsubsection*{Meça o desajuste (misadjustment)}
Para calcular o desajuste, podemos utilizar:
\begin{align*}
    M &= \frac{\xi_{\text{exc}}}{\xi_{\text{min}}} \\
    &\approx \frac{\mu \text{trace}(\mathbf{R}_{x})}{1 - \mu \text{trace}(\mathbf{R}_{x})}
\end{align*}

Utilizando o MATLAB, foi possível calcular o desajuste a partir dessa expressão foi possivel obter a Tabela~\ref{tab:desajuste}.

\begin{table}[!htp]
    \centering
    \begin{tabular}{|l|l|l|l|}
        \hline
        & $ \mu_{\text{max}}/2$ & $\mu_{\text{max}}/10$ & $\mu_{\text{max}}/50$ \\ \hline
        Téorico & -1.3846  & 2.5714 & 0.1682 \\ \hline
        Empírico & -1.3868 & 2.5342 & 0.1674 \\ \hline
    \end{tabular}
    \caption{Comparativo entre os valores de desajuste teóricos e empíricos.}
    \label{tab:desajuste}
\end{table}

\subsubsection*{Resposta em frequência do filtro FIR}
Já apresentado anteriormente nesta seção, a resposta em frequência do filtro é apresentada nas Figuras~\ref{fig:mu_2}, \ref{fig:mu_10} e \ref{fig:mu_50}. A resposta se assemelha muito do sistema que se deseja estimar para o caso de $\mu/2$. Pois o sistema tem mais adaptabilidade para seguir as mudanças do canal. Entretanto, se esse valor diminui, a identificação piora, tendo uma versão atenuadas do sistema em questão. Já para a análise no domínio do tempo, apenas o sinal do primeiro caso se assemelha, os outros dois sinais são bastante degradados e quase não se pode ver relação entre o original e o estimado.


\subsection{Equalização Adaptativa} % <-----------------------------------------------------------------------------
\todo[inline, color=yellow!30]{Organizar}

    
\begin{align*}
    H(z) = 0.5 + 1.2z^{-1} + 1.5z^{-2} + z^{-3},
\end{align*}

e deseja-se projetar um equalizar para o mesmo. A estrutura do equalizador é mostrada na Figura abaixo. Os símbolos $s(n)$ são transmitidos através de um canal e corrompidos por ruído aditivo gaussiano branco complexo $v(n)$. O sinal recebido $x(n)$ é processado pelo equalizador FIR para gerar estimativas $\overset{\sim}{s}(n - \delta)$, as quais são passados por um dispositivo decisor gerando  símbolos $\hat{s}(n - \delta)$. O equalizador possui dois modos de operação: um modo de treinamento durante o qual uma versão atrasada e  replicada da sequência de entrada é usada como o sinal de referência (desejado) e um modo dirigido por decisão no qual a saída do dispositivo de decisão substitui a sequência de referência. O sinal de entrada $s(n)$ é escolhido de uma constelação QAM (por exemplo, 4-QAM, 16-QAM, 64-QAM ou 256-QAM).



\begin{enumerate}
    
    \item Faça um programa que treine o filtro adaptativo com 500 símbolos de uma constelação 4-QAM, seguindo de uma operação dirigida por decisão de 5000 símbolos de uma constelação 16-QAM. Escolha a variância do ruído $\sigma^{2}_{v}$ de maneira que ela promova uma relação sinal ruído de 30 db na entrada do equalizador. Note que os símbolos escolhidos não têm variância unitária. Por esta razão, a a variância do ruído necessita ser ajustada adequadamente para cada uma das diferentes modulações (constelações) QAM para fornecer o nível de SNR desejado. Escolha $\delta = 15$ e o comprimento do equalizador M = 15. Mostre os gráficos da evolução temporal de $s(n)$, $x(n)$ e $\overset{\sim}{s}(n - \delta)$. Use o LMS-normalizado com um fator de passo de $\mu = 0.4$.
                
        \textcolor{red}{Solução:}

        Os resultados estão nas Figuras \ref{fig:L3Q6A1} e \ref{fig:L3Q6A2}. A evolução temporal do MSE apresenta comportamento similar aos resultados apresentados anteriormente com
        o MSE possuindo uma grande variação mesmo após a convergência. Por fim, também é possível ver na figura seguinte o impacto no filtro na filtragem de um sinal modulado por 16-QAM. É visível 
        que a estimação conseguiu separar o sinal transmitido em diferentes regiões de decisão e assim seria necessário ao fim desse processo passar esse sinal filtrado por um decisor para que fosse
        obtida uma aproximação do sinal originalmente trnasmitido. Ademais, foi implementado um simples processo de aprendizado utilizando algumas amostras de sinais 4-QAM para o treinamento do filtro com
        o objetivo de facilitar ou acelerar a convergência quando o sinal 16-QAM fosse filtrado. 
        
        Além disso, na Figura \ref{fig:L3Q6A3} é possível acompanhar a evolução temporal do sinal filtrado e do sinal 
        original para dois momentos distintos. No primeiro, temos a adaptação do filtro para as primeiras 100 amostras e previsivelmente confirmamos que o filtro possuí ainda dificuldades em acompanhar o canal 
        nas primeiras amostras. Já no segundo momento temos as últimas 100 amostras do filtro e podemos entender a partir disso que o filtro agora tem capacidade de acompanhar o canal de forma mais eficiente, embora
        ainda existam momentos onde erros ocorrem. Esses erros podem ser ocasionados pelas interferências ruidosas que provocam erros de detecção na hora de avaliar as constelações aproximadas que saem do processo de filtragem.

        \begin{figure}[!htp]
            \centering
            % \includegraphics[width=0.75\textwidth]{figs/L3Q6_A_mse.png}
            \includegraphics[width=0.5\textwidth]{example-image}
            \caption{$\text{Amostras} = 5000$, $M = 15$, $\mu = 0.4$}
            \label{fig:L3Q6A1}
        \end{figure}
        
        \begin{figure}[!htp]
            \centering
            % \includegraphics[width=0.75\textwidth]{figs/L3Q6_A_c.png}
            \includegraphics[width=0.5\textwidth]{example-image}
            \caption{$\text{Amostras} = 5000$, $M = 15$, $\mu = 0.4$}
            \label{fig:L3Q6A2}
        \end{figure}
        
        \begin{figure}[!htp]
            \centering
            % \includegraphics[width=0.75\textwidth]{figs/L3Q6_A_t.png}
            \includegraphics[width=0.5\textwidth]{example-image}
            \caption{$\text{Amostras} = 5000$, $M = 15$, $\mu = 0.4$}
            \label{fig:L3Q6A3}
        \end{figure}
        
        \clearpage

    \item Para os mesmos parâmetros do item (a), plote e compare os gráficos de evolução que seriam resultante se o equalizador fosse treinado com 150, 300 e 500 iterações. Use o LMS com um
    $\mu = 0.001$.
    
        \textcolor{red}{Solução:}

        O resultado está na Figura \ref{fig:L3Q6B}. Particularmente não pude distinguir notáveis diferenças de desempenho ao considerar diferentes tamanhos de sequências de treinamento.

            
    \begin{figure}[!htp]
        \centering
        % \includegraphics[width=0.75\textwidth]{figs/L3Q6_B_t.png}
        \includegraphics[width=0.5\textwidth]{example-image}
        \caption{$\text{Amostras} = 5000$, $M = 15$, $\mu = 0.001$}
        \label{fig:L3Q6B}
    \end{figure}

    \item Assuma agora que os dados transmitidos foram gerados de uma constelação 256-QAM ao invés de 16-QAM. Plote os gráficos da evolução do sinal na saída do equalizador quando treinado
    usando o LMS-normalizado e 500 símbolos de treinamento.					
    
        \textcolor{red}{Solução:}

        Os resultados estão nas Figuras \ref{fig:L3Q6C1} e \ref{fig:L3Q6C2}. É possível verificar na primeira figura que a evolução do MSE aconteceu apesar de uma considerável variação ao final
        da convergência. Já em relação a figura seguinte, onde transmitimos um sinal 256-QAM e utilizamos um filtro treinado por um sinal 4-QAM, é possível verificar que inicialmente o filtro demonstrou dificuldade em acompanhar
        a evolução do canal. Por fim, em um segundo momento é apresentado a evolução temporal considerando as ultimas amostras do sinal e aqui podemos visualizar que, embora ainda existam uma quantidade considerável de erros por se 
        tratar de uma modulação de ordem elevada, o filtro consegue se aproximar com um pouco mais de facilidade do sinal original. 
        
        Adicionalmente, para termos uma melhor interpretação do desempenho desse filtro seria necessário utilizar 
        uma métrica como a SER ou BER e comparar o impacto de diferentes modulações no desempenho do algoritmo de filtragem. Isso se deve pois quando utilizamos um decisor o símbolo filtrado pode ser alocado para uma região da constelação
        da modulação a qual não pertence. Entretanto, tais erros de decisão so são aparente quando utilizamos uma métrica capaz de capturar tal fenômeno.
        
        
    \begin{figure}[!htp]
        \centering
        % \includegraphics[width=0.75\textwidth]{figs/L3Q6_C_mse.png}
        \includegraphics[width=0.5\textwidth]{example-image}
        \caption{$\text{Amostras} = 5000$, $M = 15$, $\mu = 0.4$}
        \label{fig:L3Q6C1}
    \end{figure}

    \begin{figure}[!htp]
        \centering
        % \includegraphics[width=0.75\textwidth]{figs/L3Q6_C_t.png}
        \includegraphics[width=0.5\textwidth]{example-image}
        \caption{$\text{Amostras} = 5000$, $M = 15$, $\mu = 0.4$}
        \label{fig:L3Q6C2}
    \end{figure}

    \item Gerar as curvas de taxa de erro de símbolo (SER, do inglês Symbol Error Rate) versus SNR na entrada do equalizador para símbolos de constelações 4, 16, 64 e 256-QAM. Faça SNR variar
    de 5dB a 30dB.
    
        \textcolor{red}{Solução:}

        Os resultados estão na Figura \ref{fig:L3Q6D}. Assim, podemos analisar o real desempenho do filtro quando associado a um equalizador que desconhece o sinal verdadeiro. 
        É possível verificar que a medida que a ordem de modulação aumenta o desempenho sofre uma considerável piora. Existem duas principais razões para explicar esse comportamento. A primeira vem justamente 
        do tamanho das constelações digitais que a medida que a ordem aumenta tem uma menor regiao de decisão associada aos seus símbolos. Desse modo, o sinal transmitido fica sujeito a interferências de componentes 
        ruidosas do canal pois as divisórias entre as diversas regiões de decisão dos símbolos da modulação ficam menores e erros de decisão irão acontecer com maior frequência A segunda razão poderia ser explicada pelo 
        fato do filtro ter sido inicialmente treinado utilizando símbolos modulados com 4-QAM, mas provavelmente isso é insuficiente quando a ordem de modulação do sinal transmitido cresce.
        Ademais, vale ressaltar que não houve erros para SNR $= 30$ dB quando se transmitiu sinais 4-QAM, entretanto como foram utilizados apenas 1000 realizações de Monte Carlo não foi
        possível capturar erros para esse ponto específico de SNR.

        
    \begin{figure}[!htp]
        \centering
        % \includegraphics[width=0.75\textwidth]{figs/L3Q6_D_ser.png}
        \includegraphics[width=0.5\textwidth]{example-image}
        \caption{$\text{Amostras} = 5000$, $M = 15$, $\mu = 0.4$}
        \label{fig:L3Q6D}
    \end{figure}


\end{enumerate}
